\documentclass[12pt,a4paper,final]{article}
\usepackage[left=2.5cm,right=2.5cm,top=2.5cm,bottom=2.5cm]{geometry}

%% IDIOMA
\usepackage[utf8]{inputenc}
\usepackage[portuguese]{babel}

%% TRANSFORMAÇÕES ESTILO CSS
\usepackage{graphicx}

%% ESTÉTICA
\usepackage{enumerate}
\usepackage{booktabs}
\usepackage{amsmath, amsthm, amssymb, amsfonts}
\usepackage{multirow}
\usepackage[hyphens]{url}
\usepackage{subfig}

%% FONTE
\usepackage[T1]{fontenc}
%\usepackage[sc]{mathpazo} % Palatino with smallcaps
\usepackage{mathptmx}
\usepackage{eulervm} % Euler math

%% TIPOGRAFIA
\usepackage{parskip}
\usepackage[activate={true,nocompatibility},final,tracking=true,kerning=true,spacing=true,factor=1100,stretch=10,shrink=10]{microtype}

%% CODIGO
\usepackage{listings}
\usepackage{color}

\definecolor{dkgreen}{rgb}{0,0.6,0}
\definecolor{gray}{rgb}{0.5,0.5,0.5}
\definecolor{mauve}{rgb}{0.58,0,0.82}

\lstset{frame=tb,
  aboveskip=3mm,
  belowskip=3mm,
  showstringspaces=false,
  columns=flexible,
  basicstyle={\small\ttfamily},
  numbers=none,
  numberstyle=\tiny\color{gray},
  keywordstyle=\color{blue},
  commentstyle=\color{dkgreen},
  stringstyle=\color{mauve},
  breaklines=true,
  breakatwhitespace=true,
  tabsize=3
}

\title{Relatório 4 de TCC2/IC}
\author{Ly Sandro Amorim de Campos Salles\\Departamento de Física\\Universidade Federal do Paraná}
\date{\today}

\begin{document}
	\maketitle

  As seguintes atividades foram realizadas desde o último encontro:

  A releitura da tese ``Dinâmica de padrões em autômatos celulares com inércia'' do pesquisador Klaus Kramer. Dessa leitura foram feitas as seguintes observações:
  \begin{enumerate}
    \item O valor de cada célula num autômato celular pode indicar vários coisas além de um ``estado'': densidade de partículas, energia, cor de pigmento;
    \item O Autômato Celular tem três possíveis estados para cada célula (\texttt{-1}, \texttt{0} e \texttt{+1}); as células estão organizadas em uma malha quadrada; as oito celulas adjacentes formam a vizinhança de cada célula; a matriz de células é finita; células nas bordas interagem somente com as células nas bordas ou no interior da matriz; cada célula tem uma resistência à mudança cujo valor fica em $[0, 8]$, sendo $0$ equivalente a nenhuma resistência e $8$ equivalente a resistência total; existe uma chance de cada célula que tem uma vizinha no estado $0$ ser convertida para o estado $0$; ao ser atualizada, se a a célula tiver uma inércia maior que o valor absoluto da soma dos estados das células na vizinhança dela, a célula continua no mesmo estado; caso o valor da inércia da célula seja menor que o valor absoluto da soma da vizinhança, a célula muda para o estado \texttt{-1} ou \texttt{+1} dependendo se a soma é menor ou maior que $0$, respectivamente; se o valor da soma for zero, a célula continua no mesmo estado.
    \item O valor da inércia é igual para todas as células e não muda durante a simulação.
    \item Para a análise das simulações, os seguintes parâmetros foram utilizados: a \textbf{densidade de cada estado, o tamanho médio dos aglomerados, o número de aglomerados}, e o tempo de convergência para uma disposição estacionária.
    \item Aglomerados são formados por duas ou mais células adjacentes no mesmo estado.
    \item Para a contagem dos aglomerados é utilizado o \textbf{algoritmo de Hoshen-Kopelman}, que consiste em encontrar aglomerados, etiquetar os aglomerados encontrados, procurar aglomerados adjacentes, realizar a união desses aglomerados, e iterar os dois últimos passos até que não existam mais aglomerados adjacentes.
    \item A contagem de aglomerados foi implementada com o objetivo de estudar \textit{qualitativamente} a evolução da distribuição espacial dos estados do sistema.
    \item Uma configuração é considerada estacionária se a média dos estados das últimas dez gerações estiver dentro de $0.001\%$ da média dos estados dessa configuração.
    \item A \textbf{matriz inicial} é criada parcialmente de forma aleatória, sendo que a densidade do estado \texttt{0} é forçada para um valor pré-definido e o estado \texttt{+1} é forçado a ser o dominante.
    \item O \textbf{espaço de amostra} para as simulações foi escolhido pelo autor: os conjuntos de dados escolhidos tinham uma matriz inicial com tamanho médio de aglomerados entre $2.4$ e $2.8$. Isso foi feito porque ``os valores resultaram numa melhor distribuição dos dados iniciais sobre o espaço de parâmetros''.
    \item Para todos os valores de inércia, a densidade de células positivas aumentou até o estado estacionário. A relação entre densidade inicial de células positivas e densidade final de células positivas é localmente linear no intervalo $[0.33, 0.36]$.
    \item \textbf{Densidades maiores de células no estado positivo resultaram na diminuição de aglomerados positivos.}
    \item É verificada a evolução de \textbf{uma mesma matriz inicial para vários valores de inércia} em $[0,8]$.
    \item É feita uma \textbf{análise do comportamento de todas as possíveis vizinhanças de uma célula}, verificando qual a porcentagem das vizinhanças que convertiam o estado da célula e qual a porcentagem das vizinhanças que não mudavam o estado da célula, para cada valor de inércia em $[0,8]$.
    \item É feito um estudo sobre quais as \textbf{condições necessárias para a existência de fronteiras} (fronteira é o conjunto das células cujo número de celulas vizinhas no estado \texttt{-1} é igual ao número de células vizinhas no estado \texttt{+1}).
    \item É estudado, para larguras de matriz iguais a $20$ $40$ $60$ $100$ e $200$ células, qual a \textbf{influência do tamanho da matriz sobre a dinâmica do sistema}. Para isso foi tido como objetivo gerar \textbf{mil matrizes diferentes para cada valor pré-estabelecido} de densidade de células no estado \texttt{+1}. Esses valores pré-estabelecidos são todos próximos e maiores que $1/3$.
    \item É concluído que a \textbf{densidade final de células no estado \texttt{+1} independe do tamanho do sistema}.
    \item Para \textbf{maiores larguras de matriz} foi observado que o \textbf{tamanho médio dos aglomerados aumentou}.
    \item É explicado que o comportamento anômalo para quando a \textbf{inércia é igual à metade do número de células na vizinhança} de cada célula é devido ao fato de as \textbf{mudanças de estado serem menos frequêntes} quando isso acontece e as \textbf{células das bordas nunca atualizarem} nesse caso. Com isso, como as bordas têm muita influência em \textbf{matrizes pequenas}, o comportamento anômalo é criado.
    \item \textbf{Quando a inércia aumenta, o número de aglomerados também aumenta}, pois as \textbf{células com mesmo estado ficam menos propensas a se unirem}, o que faria o número de aglomerados dinimuir. \texttt{<Explica a subida após o poço de potencial do nosso Autômato Celular: a inclinação do número de aglomerados \textit{versus} estado médio aumenta quando o valor de $q$ aumenta pois fica mais difícil de as células se unirem. Enquanto isso, a grande queda do poço de potencial é devido a uma crescente melhora na tendência de as células se unirem. Essas observações estão de acordo com o termo \textit{afinidade} estabelecido nos estudos anteriores>}
    \item Para matrizes não-convergentes foram procuradas convergências periódicas.
    \item Foi observado que o \textbf{comportamento dos parâmetros} analisados é muito \textbf{parecido entre as matrizes que convergem e as que não convergem}.
    \item Ao serem considerados valores diferentes de inércia para cada estado, foi utilizada uma matriz cuja maioria das células estava no estado \texttt{0}. Foi feita uma analogia na qual os estados \texttt{-1} e \texttt{+1} representavam biomas diferentes e o estado \texttt{0} representava uma região de desmatamento.
    \item Sobre um gráfico (Figura 5.4-(c) da Tese, Página 60) que apresentava um pico em formato de potencial de Lennard-Jones negativo, o autor explicou que isso aconteceu porque, enquanto o número de aglomerados aumenta, a densidade do estado \texttt{+1} continua diminuindo, chegando um ponto em que o número de aglomerados se estabiliza (pico) e logo em seguida ocorre a queda do número de aglomerados. A Figura 5.10 (página 66 da tese) apresenta mais gráficos com esse comportamento. \texttt{<Ele considerou um eixo x (densidade do estado +1) invertido nesta explicação>}
    \item Ao ser considerado um modelo no qual a inércia é a menor para o estado predominante e é a maior para o estado menos dominante, foi observado que a densidade dos estados \texttt{-1} e \texttt{+1} ficam praticamente sobrepostas no tempo, alternando em altura regularmente.
    \item Quando foi considerada uma inércia que variava periodicamente com o tempo, foi observado um padrão nos parâmetros parecido com o obtido nos casos anteriores, com a exceção do fato de os parâmetros sofrerem oscilações sinusoidais.
    \item Ao estudar a evolução de padrões geométricos, foi observado que todos eles se expandem, mantendo o formato com inércia igual $0$ e tendendo a círculos para inércia igual a $2$.
    \item No estudo de ecótonos (região de sobrevivência de um estado entre a fronteira competitiva de outros dois estados), foi estudada a evolução de uma única matriz inicial considerando vários parâmetros diferentes. O objetivo foi mostrar que a emergência de ecótonos pode ser explicada através de um autômato celular. Comportamentos semelhantes aos reais foram observados para valores pequenos de inércia (0, 1 e 2). Quando o valor da inércia foi definido como $3$, houve a situação de o ecótono tomar conta de um dos outros dois biomas, quebrando o equilíbrio. Foi concluído que a única possibilidade de haver a expansão do estado \texttt{0} é se houver uma chance não-nula de qualquer célula nos outros estados com pelo menos uma célula vizinha no estado \texttt{0} mudar para o estado \texttt{0}.
  \end{enumerate}
  
  Foi feita a leitura da Seção 1 (``An outline of basic ideas'') do Capítulo 1 (``The Foundations for a New Kind of Science'') do Livro ``A New Kind Of Science'' do autor Stephen Wolfram. Adicionalmente, lendo trechos de outras partes do livro foi possível perceber que estes são os axiomas utilizados pelo Stephen Wolfram:
  \begin{enumerate}
    \item É possível criar sistemas complexos a partir de regras simples.
    \item Princípio da equivalência computacional: para todo comportamento que não é obviamente simples  existe uma computação correspondente equivalente sofisticada.
    \item É possível descobrir como um sistema vai se comportar executando experimentos e observando o que acontece. O sucesso das ciências aconteceu ao serem encontradas fórmulas matemáticas que previam os resultados desses experimentos, criando um atalho no trabalho que o sistema realiza por si próprio. Esses exemplos demonstram que, apesar de o princípio da equivalência computacional garantir uma resposta mais complexa, soluções mais simples podem existir.
    \item Apesar da possibilidade de sistemas simplificáveis, podem existir sistemas irredutíveis computacionalmente, ou seja, cuja única forma de prever o resultado do experimento é executando o experimento.
  \end{enumerate}
  Este último axioma é o que causou Stephen a desenvolver a teoria de autômatos celulares descrita no livro dele.

  O relatório parcial de Iniciação Científica foi escrito e enviado ao sistema Siga UFPR.

  Foi feita uma conjectura sobre como controlar o formato do poço de pontencial do gráfico de afinidade \textit{versus} $q$: se quisermos aumentar a abertura/largura do poço de potencial podemos aumentar o número de células vizinhas a cada célula pois isso aumentaria a tendência de as células se aglomerarem, mantendo negativa por mais tempo a inclinação do atrator dos gráficos número de aglomerados \textit{versus} $q$; analogamente podemos ``fechar'' o poço de potencial diminuindo o número de células vizinhas a cada célula. Portanto, controlando o número de células vizinhas a cada célula podemos fazer uma ``afinação'' (\textit{fine tuning}) do gráfico gerado pelo autômato celular para encaixar perfeitamente no gráfico do potencial de Lennard-Jones, a menos de translação e escala.

  Para os próximos dias, estas serão as tarefas realizadas:
	\begin{enumerate}
		\item A finalização do programa de simulações de autômatos celulares.
		\item Verificação da sopreposição da curva do potencial de Lennard-Jones no gráfico de afinidade em função de $q$.
		\item A leitura do Capítulo 4 (``Nonlinear Oscillations and Chaos'') do livro ``Classical Dynamics of Particles and Systems'' dos autores Stephen T. Thornton e Jerry B. Marion.
	\end{enumerate}

\end{document}
