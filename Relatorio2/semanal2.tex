\documentclass[12pt,a4paper,final]{article}
\usepackage[left=2.5cm,right=2.5cm,top=2.5cm,bottom=2.5cm]{geometry}

%% IDIOMA
\usepackage[utf8]{inputenc}
\usepackage[portuguese]{babel}

%% TRANSFORMAÇÕES ESTILO CSS
\usepackage{graphicx}

%% ESTÉTICA
\usepackage{enumerate}
\usepackage{booktabs}
\usepackage{amsmath, amsthm, amssymb, amsfonts}
\usepackage{multirow}
\usepackage[hyphens]{url}
\usepackage{subfig}

%% FONTE
\usepackage[T1]{fontenc}
%\usepackage[sc]{mathpazo} % Palatino with smallcaps
\usepackage{mathptmx}
\usepackage{eulervm} % Euler math

%% TIPOGRAFIA
\usepackage{parskip}
\usepackage[activate={true,nocompatibility},final,tracking=true,kerning=true,spacing=true,factor=1100,stretch=10,shrink=10]{microtype}

%% CODIGO
\usepackage{listings}
\usepackage{color}

\definecolor{dkgreen}{rgb}{0,0.6,0}
\definecolor{gray}{rgb}{0.5,0.5,0.5}
\definecolor{mauve}{rgb}{0.58,0,0.82}

\lstset{frame=tb,
  aboveskip=3mm,
  belowskip=3mm,
  showstringspaces=false,
  columns=flexible,
  basicstyle={\small\ttfamily},
  numbers=none,
  numberstyle=\tiny\color{gray},
  keywordstyle=\color{blue},
  commentstyle=\color{dkgreen},
  stringstyle=\color{mauve},
  breaklines=true,
  breakatwhitespace=true,
  tabsize=3
}

\title{Relatório 2 de TCC2/IC}
\author{Ly Sandro Amorim de Campos Salles\\Departamento de Física\\Universidade Federal do Paraná}
\date{\today}

\begin{document}
	\maketitle

	Desde o último encontro, o Capítulo 16 (``Structures, Unions and Enumerations'') do Livro ``C Programming: A Modern Approach'' foi lido.
  O Capítulo 17 (``Advanced Uses of Pointers'') do mesmo livro está sendo lido. Com a finalização da leitura desse Capítulo, estarão finalizados os estudos na área de programação necessários para o desenvolvimento da nova versão do programa que realiza as simulações de autômatos celulares. O progresso no desenvolvimento desta nova versão, chamada de \textit{Cellular Automata Explorer}, pode ser acompanhado em tempo real pelo endereço \url{https://github.com/Ly54ndr0/CellularAutomataExplorer}.
	
	As próximas atividades programadas são:
	\begin{enumerate}
		\item A releitura do artigo ``Adjustment and social choice'' dos pesquisadores Gérard Weisbuch e Dietrich Stauffer; 
		\item A releitura da tese ``Dinâmica de padrões em autômatos celulares com inércia'' do pesquisador Klaus Kramer;
		\item A leitura do Capítulo 4 (``Nonlinear Oscillations and Chaos'') do livro ``Classical Dynamics of Particles and Systems'' dos autores Stephen T. Thornton e Jerry B. Marion;
		\item O desenvolvimento do relatório parcial de Iniciação Científica;
		\item A finalização do programa de simulações de autômatos celulares.
	\end{enumerate}

\end{document}
